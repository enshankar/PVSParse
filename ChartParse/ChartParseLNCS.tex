\documentclass[letter]{llncs}
\usepackage{booktabs}
\usepackage{subcaption}
%\usepackage{incgraph}
%\usepackage{lmodern}
\usepackage{amsthm}
\usepackage{stmaryrd}
\usepackage{mathrsfs}
%\usepackage{textcomp}
\usepackage{mdframed}
%\usepackage{titlesec}
\usepackage{wrapfig}
\usepackage{bussproofs}
\usepackage{balance}
\usepackage{relative}
\newenvironment{bprooftree}
  {\leavevmode\hbox\bgroup}
  {\DisplayProof\egroup}
%\usepackage{relative}

\usepackage{listings}
%% Gestion des codes sources
\lstdefinelanguage{PVS}{
   morekeywords={AND, CONJECTURE, FACT, LET, TABLE, ANDTHEN, CONTAINING, FALSE, LIBRARY, THEN, ARRAY, CONVERSION, FORALL, MACRO, THEOREM, ASSUMING, CONVERSION+, FORMULA, MEASURE, THEORY, ASSUMPTION, CONVERSION-, FROM, NONEMPTY, TYPE, TRUE, AUTO, REWRITE, COROLLARY, FUNCTION, NOT, TYPE, AUTO, REWRITE+, DATATYPE, HAS, TYPE, O, TYPE+, AUTO, REWRITE-, ELSE, IF, OBLIGATION, VAR, AXIOM, ELSIF, IFF, OF, WHEN, BEGIN, END, IMPLIES, OR, WHERE, BUT, ENDASSUMING, IMPORTING, ORELSE, WITH, BY, ENDCASES, IN, POSTULATE, XOR, CASES, ENDCOND, INDUCTIVE, PROPOSITION, CHALLENGE, ENDIF, JUDGEMENT, RECURSIVE, CLAIM, ENDTABLE, LAMBDA, SUBLEMMA, CLOSURE, EXISTS, LAW, SUBTYPES, COND, EXPORTING, LEMMA, SUBTYPE, OF},
    sensitive=false, % keywords are not case-sensitive
    morecomment=[l]{\%}, % l is for line comment
    morestring=[b]" % defines that strings are enclosed in double quotes
}
\definecolor{gainsboro}{rgb}{0.86, 0.86, 0.86}
\lstset{keepspaces=true}
\lstset{columns=fixed}
\lstset{linewidth=.99\linewidth}
\lstset {
	basicstyle=\small\smaller\ttfamily,
	keywordstyle=\color{purple},
	commentstyle=\color{red},
	breaklines,
	breakindent=0.7em,
	backgroundcolor=\color{gainsboro},
    %numbers=left,
    tabsize=2,
    %stepnumber=1,
	frame=single,
    inputencoding=utf8,
    extendedchars=true,
	literate={\ \ }{{\ }}1 {ε}{{$\epsilon$}}1 {λ}{{$\lambda$}}1 {∀}{{$\forall$}}1 {⇒}{{$\Rightarrow$ }}1 {∧}{{$\wedge$ }}1 {∨}{{$\vee$ }}1 {Δ}{{$\Delta$}}1 {⇔}{{ $\iff$ }}1 {ω}{{$\omega$ }}1  {⊥}{{$\bot$ }}1 {⊤}{{$\top$ }}1
}
\newcommand{\lstin}[1]{\lstinline[language=PVS]{#1}}



\newcommand{\prodnInternal}{\leftarrow}
\newcommand{\prodn}[2]{#1 \prodnInternal #2}
\usepackage{tikz}
\usetikzlibrary{tikzmark}
\newcommand{\myparagraph}[1]{}

%% Title information
\title{Formal Development of a Top-Down Chart Parser for Parsing Expression Grammars}
% \titlenote{This work was supported by the National Institute of Aerospace
%  Award C18-201097-SRI, NSF Grant
%  SHF-1817204,  and DARPA under agreement number HR001119C0075. 
%  The views and conclusions contained herein are those of the authors and
%  should not be interpreted as necessarily representing the official
%  policies or endorsements, either expressed or implied, of NASA, NSF, DARPA, Ecole Polytechnique, or the
%  U.S. Government.  We thank the anonymous referees for their detailed comments and 
%  constructive feedback.
% }
%% Author information
%% Contents and number of authors suppressed with 'anonymous'.
%% Each author should be introduced by \author, followed by
%% \authornote (optional), \orcid (optional), \affiliation, and
%% \email.
%% An author may have multiple affiliations and/or emails; repeat the
%% appropriate command.
%% Many elements are not rendered, but should be provided for metadata
%% extraction tools.

%% Author with single affiliation.

\author{Zephyr S. Lucas\inst{1} \and Natarajan Shankar\inst{2}} 
 \institute{
{Department of Computer Science}\\
{Dartmouth College}\\
{6211 Sudikoff Laboratory}\\
{Hanover}
{NH}
{03755-3510}
{USA}\\
{Zephyr.S.Lucas.GR@dartmouth.edu}}


\institute{
{Computer Science Laboratory}\\
{SRI International}\\
{333 Ravenswood Avenue}\\
{Menlo Park}
{CA}
{94025}
{USA}\\
{shankar@csl.sri.com}}

\begin{document}

\maketitle

\begin{abstract}
  Formalization is often seen as an exclamation point 
   certifying the correctness of a mathematical development.  However, in
  the context of an interactive proof assistant, formalization is the medium for a
  creative dialogue that identifies gaps, tags errors, and refines the
  definitions, claims, and proofs to enhance the correctness, clarity,
  elegance, robustness,
  and reusability of mathematical content.  With this goal in
  mind, we explore the formalization of a top-down chart parser for
  parsing expression grammars (PEGs) using SRI's Prototype
  Verification System (PVS).  The parser, which supports dynamic loop
  detection, operates on a chart-recording data structure that also embeds the
  parsing stack.  Since the parsing algorithm is tail-recursive and
  operates on fixed storage, it is hardware-friendly.  The invariants
  needed to establish the correctness of the chart parser are complex
  and delicate.  They were derived through an iterative proof
  exploration process.  The generated proof obligations and proof
  attempts led to a number of significant corrections and improvements
  to the formal development.
  % We examine the interaction between the
  % language, type system, and proof automation/interaction in PVS in
  % the context of creating, revising, and restructuring formal content.
\end{abstract}



%% Keywords
%% comma separated list
\keywords{PVS, PEG grammar, chart parsing, verified parser, top-down parsing}  %% \keywords are mandatory in final camera-ready submission


%% \maketitle
%% Note: \maketitle command must come after title commands, author
%% commands, abstract environment, Computing Classification System
%% environment and commands, and keywords command.



%% Acknowledgments
% \begin{acks} This work was supported by NSF Grant
% CSR-EHCS(CPS)-0834810, NASA Cooperative Agreement NNA10DE73C, and by
% DARPA under agreement number FA8750-12-C-02&84 and  FA8750-16-C-0043. 
% The views and conclusions contained herein are those of the authors and
% should not be interpreted as necessarily representing the official
% policies or endorsements, either expressed or implied, of NSF, NASA, DARPA or the
% U.S. Government.
% \end{acks}


\section{Introduction}
\label{sec:intro}


Parsing is a fundamental operation for transforming concrete
representations of data such as documents and program text into their
abstract machine-representable counterparts~\cite{GruneJacobs}.  Even though parsing is
one of the most developed fields of computing, a number of significant
computing vulnerabilities have been attributable to parsing
errors. For example,  parsing errors were detected in the unverified
front-end of the verified CompCert
compiler~\cite{csmith}, and security vulnerabilities~\cite{DBLP:journals/usenix-login/BratusHHLMPS17} such as Heartbleed~\cite{carvalho2014heartbleed}  
and SIGRed~\cite{SIGRed} stem from improper input
validation attributable to parsing errors. Such shortcomings are
especially lamentable given that language theory and parsing are
amenable to rigorous formalization.   We present the formalization of a
top-down chart parsing interpreter for PEG grammars developed using
SRI's Prototype Verification System (PVS)~\cite{Owre95:prolegomena}.  Equal in importance to
the correctness guarantees derived from the formalization
is the iterative proof development process by which formalization is achieved.
We  describe how through an interactive dialogue, a proof
assistant accelerates the construction of a correct, robust, and
reusable formalization.


% Parsing is a fundamental operation for transforming concrete representations of data such as a document or a program text into its abstract machine-representable counterpart. The Chomsky hierarchy of grammars:

%Parsing is one of the most developed fields of computing, and yet a
%number of significant computing vulnerabilities are attributable to
%parsing errors.  Language theory and parsing are also amenable to
%rigorous formalization but there are only a few examples of verified
%parser developments in the literature.  Chart parsing is a powerful
%framework that supports both top-down and bottom-up views of parsing.
%We present the formalization of a top-down chart parsing interpreter
%developed using SRI's Prototype Verification System.  The
%parser employs a table mapping each position in the input string and
%nonterminal to an entry that represents the parse result for the
%nonterminal at the given position.  Our parser is presented as a
%tail-recursive state machine that can be easily mapped to a hardware
%implementation.  The invariants associated with the parse table are
%complex since the table encapsulates a web of relationships that are
%impacted by an update to even a single entry.  The formalization was
%greatly aided by the dialogue with an interactive proof assistant.  We
%discuss how such a dialogue helps accelerate the construction of a
%formalization that is correct, robust, and reusable.
%
%Parsing is a fundamental operation for transforming concrete
%representations of data such as a document or a program text into its
%abstract machine-representable counterpart.  Parsing errors are critical
%since they can lead to semantic errors, e.g., in the CompCert compiler~\cite{csmith}, 
%and numerous security vulnerabilities such as those caused by
%\emph{improper input validation}~\cite{DBLP:journals/usenix-login/BratusHHLMPS17}, e.g., Heartbleed, SigRed.
%
% Parsing is a fundamental operation for natural and artificial
% languages that checks and processes input in concrete syntax into an
% unambiguous machine-friendly representation. 
% The Chomsky hierarchy of grammars:

%  \begin{eqnarray*}
% \text{regular} &\sqsubset  &\\
%  		\text{context-free} &\sqsubset  &\\
%  		\text{context-sensitive} &\sqsubset &\\
%  		\text{recursively enumerable}
%  \end{eqnarray*}
%  has dominated the discussion of parsing.
%  Such grammars can be ambiguous in admitting multiple distinct parse trees for the
%  same input string.
% Within parsing, much of the attention has focused
% on deterministic context-free languages since they admit efficient
% parsing. Parsing is usually done by
%  applying a lexer to extract a stream
%   of tokens,
%   parsing this stream according to a grammar specification to produce a
%   parse tree, and
%  applying semantic actions to evaluate the parse tree.


 Parsers~\cite{GruneJacobs} can operate
 \begin{itemize}
 \item \emph{Top-down}, by consuming input token strings
    matching a given grammar category, e.g., recursive descent and
    $LL(k)$ parsers, and Earley parsing for arbitrary context-free grammars,
    or 
 \item \emph{Bottom-up}, where the
   contiguous tokens are grouped into grammar categories, e.g.,
   precedence parsers, LR(k) (and variants SLR,
 LALR, CLR), as well as GLR and CYK parsers for arbitrary context-free
 grammars. 
 \end{itemize}

% A top-down recursive descent parser is just a mutually recursive program with a
% parsing function for each nonterminal.  An $LL(k)$ (Left-to-right scan of the input,
% Leftmost derivation with $k$ token look-ahead) parser is another example of a
% top-down parsing scheme.  $LL(k)$ parsing starts with the construction
% of a lookup table associating production rules from the given grammar
% with nonterminals and look-ahead strings.  $LL(k)$ parsing employs a stack of symbols
% (terminals and nonterminals) together with the lookup table to either
% move the cursor representing the parse position in the input string
% when the token at the cursor matches the top symbol in the stack, or
% looking up the matching production for a nonterminal at the top of the
% stack against the $k$ leading tokens at the cursor, identifying the
% matching production and replacing the top nonterminal with the symbols
% (in reverse order) from the right-hand side of the production.  Not
% all grammars support consistent $LL(k)$ lookup tables.  Also, left
% recursion poses a challenge for top-down parsing since it can lead to a nonterminating parse.

% Bottom-up parsers work by processing the input token stream and
% constructing the parse tree from a forest generated by the leaves upwards.
% Examples of bottom-up parsers  include precedence parsers, LR(k) (and variants SLR,
% LALR, CLR), and GLR  and CYK parsers for arbitrary context-free
% grammars.  Left recursion is a challenge for some but not all bottom-up parsers.
% The popular class of shift-reduce parsers use a
% stack to extract matching nonterminals for a substring by applying
% one of 
% \begin{itemize}
% \item A \emph{shift} step which moves a token from the input onto the stack
% \item A \emph{reduce} step which replaces the top few symbols in the stack
%   with a matching nonterminal according to a grammar production rule.  
% \end{itemize}

Parsing algorithms like the Earley and CYK algorithms are examples of \emph{chart
parsers}~\cite{earley1970efficient,kay1986algorithm,pereira1983parsing} based on dynamic programming where the goal is to
progressively label larger and larger spans of the input string with
terminals and nonterminals based on the labels of contiguous subspans.
Earley parsing takes a top-down approach whereas the CYK and Valiant
algorithms take a bottom-up approach to parsing.

Our chart parser employs a table or a \emph{scaffold} mapping each
position in the input string and non-terminal to an entry that
represents the parse result for the non-terminal at the given
position. Our chart parser is a tail-recursive state machine that
operates on fixed storage so that it can be easily mapped to a
hardware implementation. With respect to formalization, the invariants
associated with the parse table are complex since the table
encapsulates a web of relationships that are impacted by an update to
even a single entry. Using PVS as an interactive proof assistant
greatly assisted in managing this complexity.  More strongly, it 
would be extremely hard to construct or understand such proofs
without the aid of a proof assistant. 


In the present work, we focus on a class of grammars called Parsing
Expression Grammars (PEGs) introduced by Ford~\cite{DBLP:conf/popl/Ford04}.  These grammars are
unambiguous and efficient for both top-down and bottom-up parsing.
PEGs replace the unordered choice operation $e_1 ~|~ e_2$ in CFGs with
an ordered choice operation $e_1/e_2$\@.  In addition to concatenation
$e_1 e_2$, the grammar supports an check operation $\& e$ that looks
ahead for a match to $e$ without consuming any tokens, and negation
$!e$ that succeeds consuming no tokens only when a match against $e$
fails.   PEG grammars are a good match for  recursive descent
parsers.  Parser combinators can be defined for each PEG construct that
combine parsers for the grammar sub-expressions into a parser for the
expression.  PEG grammars can be \emph{ill-formed} and contain loops, for
example in the case of left recursion where, for example, a
nonterminal $n$ is mapped to the expression $n/n'$\@.  PEG grammars can
be checked for \emph{well-formedness} through a static analysis, where
well-formed grammars always yield parses that terminate in success or
failure, however such an analysis can miss cases where the grammar
is in fact loop-free.  

Prior research has explored the formalization of PEG parsing
where the grammars are conservatively analyzed for termination using recursive
descent parsers~\cite{trx,blaudeau2020verified}\@.  Top-down parsers
that have been verified include the CakeML parser~\cite{DBLP:conf/popl/KumarMNO14},
an LL(1) parser~\cite{DBLP:conf/itp/LasserCFR19}, an SLR parser generator~\cite{conf/esop/BarthwalN09}, and a recursive descent parser for context-free languages~\cite{conf/cpp/Ridge11}\@.  We believe that our work is the first to address the
verification of an Earley-style chart parsing algorithm.  We allow ill-formed
grammars but extend the parser to dynamically check for loops.  
In this more general setting, it is possible to demonstrate that the
parser is well-behaved on well-formed grammars while still admitting potentially
ill-formed grammars.  

We also formalize the parser interpreter as a state machine by
internalizing the stack. The parser is centered around a table
$A$ where for each position $i$ on the input string, there is
an array $A(i)$ such that for each nonterminal $n$, $A(i)(n)$ is a
table entry.  Initially, all the table entries $A(i)(n)$ are
\emph{pending} indicating that parsing has not yet been initiated at
any string position $i$ for any nonterminal $n$\@.  Once the root
query has been launched for non-terminal $r$ at position $0$, we place
a pair representing $(0, r)$ on the stack.  The production
$\prodn{r}{e}$ corresponding to the nonterminal $r$ is evaluated, and
parsing is continued through a case analysis on $e$.  This can lead to
pushing more position/nonterminal pairs on to the stack or to \emph{finalized}
entries representing failed, loopy, or successful parses.  When a parse query is
initiated on a position/nonterminal pair that is already on the stack,
a loop is signaled, and this loop is propagated back through the stack
to the root query.

We arrived at the above algorithm in a series of steps.  We first
started by formalizing the \emph{scaffolding automata} algorithm due
to Loff, Moreira, and Reis~\cite{loff2020computational}.  This is a
bottom-up, right-to-left algorithm that fills in the scaffold from the
final position backwards.  In each iteration, it fills out the entries
for a given position for each nonterminal.  It is possible to sort the
nonterminals so that the sub-queries for each nonterminal at a given
position have already been processed.  This variant would be quite
straightforward to verify since the sequencing of the operations
ensures that each entry is consistent with the entries corresponding
to the sub-queries.  Since it is challenging to write a grammar that
passes a termination check, we allowed the nonterminals to be
unordered.  This introduces the possibility of loops.  For example, a
grammar with $\prodn{S}{R S}$, where $R$ happens to succeed without
consuming any tokens.  We tried an alternative that involved using a
stack to process any as yet unprocessed sub-queries.  The use of the
stack complicated the picture.  We decided that if we were going to
use a stack, then we might as well employ a top-down algorithm.  We
then formalized a top-down algorithm with an explicit stack argument,
and soon noticed that the stack could be embedded within the scaffold
itself.  We then wrote out a full state machine and a few days of
dialogue with the proof assistant yielded the invariants for the
parsing algorithm.  We then noticed that the flat state machine could
be refactored to make the proofs less repetitive and more compact.
This required additional strengthenings of the invariants suggested by
a couple more days of dialogue with the proof assistant.  Many of the
definitions and invariants would have been painfully difficult to
derive without the benefit of the interactive dialogue supported by
powerful automation.  Many of the proofs employ a heavy amount of
automation even when it is obvious from experience that the automation
will succeed.  Others need to be handled more delicately since they
might fail and we as users need to understand the root cause of the
failure.

The parsing algorithm we formalize is attractive since it captures
top-down PEG parsing as a state machine with a single statically
allocated data structure.  This makes it hardware-friendly.  We added
dynamic loop detection in order to support a more liberal class of 
PEGs that admit cyclic dependencies and to address a novel verification
challenge.  If the grammar does not contain cyclic dependencies, the
loop detection can be turned off.  Our basic
algorithm can also support certain optimizations for speeding up parsing.
The chart parsing algorithm presents several challenges for verification.
The key result we want at the end is that the parsing operation
yields a consistent scaffold yielding a chart labeling the root query.  Defining a consistent
scaffold is itself quite challenging.  The conditions on partially
completed scaffold have to be characterized in such a way that each step
of the parsing algorithm preserves this condition as an invariant, and
the resulting completed chart is consistent.  The final challenge is
structuring the proof in order to make it easy to demonstrate that the
invariant is preserved.  



% >>>>> Following has been moved up >>>
%Despite its importance as a semantic bridge and a
%critical access point for security attacks, formal aspects of parsing have not
%been deeply analyzed.
%A few parsing algorithms have been verified.


\section{A Top-Down Chart Parser for PEGs}\label{sec:PEG}

%We employ a Chomsky Normal Form
%representation of PEG grammars where the right hand sides of
%productions are flat, i.e., contain no nesting of PEG operations.  Any
%PEG grammar can easily be represented in Chomsky Normal Form, whereas
%the corresponding transformation for CFGs is more complicated.
%Let $A$, $B$, $C$ range over nonterminal symbols, and $M$, $N$ range over
%terminal operations including the character $c$,
%the any operation $any(p)$, where $p$ is a predicate over characters, and
%$\epsilon$ representing the empty string.  The Chomsky Normal Form
%grammar consists of productions of the form $A \leftarrow e$, where
%$e$ is of one of the forms 
%\begin{center}
%\begin{tabular}{{|l|c|}}\hline
%  Empty &   $\epsilon$ \\\hline
%  Failure & $f$ \\\hline
%  Any & $any(p)$ \\\hline
%  Terminal & $c$ \\\hline
%  Concatenation & $M N$\\\hline
%  Ordered Choice & $M/N$ \\\hline
%  Check & $\& M$ \\\hline
%  Negation & $!M$ \\\hline
%\end{tabular}
%\end{center}
%
%Central to our parser is a table data structure similar to those used by
%Earley and CYK parsers.  The table, which we refer to as a scaffold following
%For parsing a string $s$ of length $L$,
%The scaffold consists of entries of the form $a_{in}$ for string position $i$, $0\leq i\leq L$,
%and nonterminals $n$\@.
%
%Each entry $a_{in}$ is either \textit{pending}, \textit{fail}, \textit{loop}, or \textit{good(height, span)}.  Since we embed the stack in the scaffold,
%we also allow entries of the form \texttt{push(position, nonterminal)}
%Initially, all entries in the scaffold are pending.  
%$$
%\begin{array}{|r||c|c|c|c|c|}\hline
%  & 0 & \ldots & i & \ldots & L \\\hline\hline
%  n0 & a_{00}& \ldots & a_{i0}\tikzmark{a} & \ldots & a_{L0}\\\hline
%  \vdots & \vdots & \vdots & \vdots & \vdots & \vdots\\\hline
%  nj & \ldots & \ldots & a_{ij}\tikzmark{b} & \ldots & a_{Lj}\\\hline
%  \vdots & \vdots & \vdots & \vdots & \tikzmark{c}\vdots & \vdots\\\hline  
%  nN & a_{0N}& \ldots & a_{iN} & \ldots & a_{LN}\\\hline
%\end{array}
%$$
%\begin{tikzpicture}[overlay, remember picture, yshift=.25\baselineskip, shorten >=.5pt, shorten <=.5pt]
%    \draw [->] ({pic cs:b}) [bend right] to ({pic cs:a});
%    \draw [->] ([yshift=.75pt]{pic cs:b}) -- ({pic cs:c});
%  \end{tikzpicture}
%The stack can be represented within the scaffold itself --- hardware-friendly. 



We employ a Chomsky Normal Form representation of PEG grammars where the right hand sides of productions are flat, i.e., contain no nesting of PEG operations.  By definition, any PEG grammar can easily be represented in Chomsky Normal Form, whereas the corresponding transformation for CFGs is more complicated.  PEG normal form requires each production to be of the form $\prodn{n_{j}}{e}$, where $n_{j}$ is a non-terminal, and $e$ is an expression of one of the eight types shown in Table~\ref{table:NormalForm}.

\begin{table}[htbp]
\begin{tabular}{{|l|c|}}\hline
  Empty &   $\epsilon$ \\\hline
  Failure & $f$ \\\hline
  Any & $any(p)$ \\\hline
  Terminal & $c$ \\\hline
  Concatenation & $n_{j} n_{k}$\\\hline
  Ordered Choice & $n_{j}/n_{k}$ \\\hline
  Check & $\& n_{j}$ \\\hline
  Negation & $!n_{j}$ \\\hline
\end{tabular}
\caption{\small Each row in contains one of the set of fundamental PEG normal-form operations we implemented.  Here $n_{j}$, $n_{k}$ are non-terminals, $c$ is a character in our character space, and $p$ is a predicate over our character space.}
\label{table:NormalForm}
\end{table}

 A generalized version of the scaffold data structure is shown in Figure~\ref{fig:GenericScaffold}.

 \begin{figure}[h!]
 \[
 \begin{array}{|r||c|c|c|c|c|c|}\hline
   & s_{0} & \ldots & s_{i} & \ldots & s_{L - 1} & \epsilon \\\hline\hline
   n_{0} & a_{00}& \ldots & a_{i0}\tikzmark{a} & \ldots & a_{(L - 1)0} & a_{L0}\\\hline
   \vdots & \vdots & \ddots & \vdots & \ddots & \vdots& \vdots \\\hline
   n_{j} & a_{0j} & \ldots & a_{ij}\tikzmark{b} & \ldots & a_{(L - 1)j} & a_{Lj} \\\hline
   \vdots & \vdots & \ddots & \vdots & \tikzmark{c}\ddots & \vdots & \vdots \\\hline  
   n_{N - 1} & a_{0(N - 1)}& \ldots & a_{i(N - 1)} & \ldots & a_{(L - 1)(N - 1)} & a_{L(N - 1)} \\\hline
 \end{array}
 \]
 \begin{tikzpicture}[overlay, remember picture, yshift=.25\baselineskip, shorten >=.5pt, shorten <=.5pt]
     \draw [->] ({pic cs:b}) [bend right] to ({pic cs:a});
     \draw [->] ([yshift=.75pt]{pic cs:b}) -- ({pic cs:c});
   \end{tikzpicture}
 \vspace*{-4mm}
 \caption{\small A generic scaffold for a PEG consisting of $N$ productions (the rows) over an input token string $s$ of length $L$ (the columns)}
 \label{fig:GenericScaffold}
 \end{figure}

 As seen in Figure~\ref{fig:GenericScaffold}, the rows of the scaffold are determined by the non-terminals in the normal form of the grammar describing the language to be parsed. The columns of the scaffold are determined by each character in the input string in left-to-right order. A given PEG grammar with normal-form productions (i.e. non-terminals) of cardinality $N$ and input string $s$ of length $L$ determines a statically-sized scaffold of size $N$ by $(L + 1)$. The size is $L + 1$ to account for an implicit empty $\epsilon$ that terminates the input string.  As described in Section~\ref{sec:intro}, our chart-parsing algorithm initializes each scaffold entry $a_{in}$ with the value \texttt{pending}. The algorithm starts with entry $a_{00}$, and proceeds in a recursively top-down manner, updating each entry $a_{in}$.


Central to our parser is a table data structure,  the scaffold, which is
similar to the tables used by Earley and CYK parsers.  The scaffold is a
two-dimensional array containing entries $a_{in}$ where $i$ is a position
in the input string $s$ with $0\leq i\leq |s|$, and $n$ is a nonterminal
from the given grammar.  
Each entry $a_{in}$ will have the value {\tt pending} (represented by a blank in Figures \ref{scaffold:matchingParenthesis} to \ref{scaffold:loop}, {\tt fail(height)} (represented by $f$), {\tt loop} (represented by $l$), {\tt good(height, span)} (represented by $g_{\texttt{span}}$,  or
{\tt push(position, nonterminal)} (abbreviated as $p_{in})$\@.
The {\tt push} value represents the recursive application of that production, thus embedding the representation of the recursion stack into the statically-sized scaffold data structure itself; a more hardware friendly design. Loops are detected when the algorithm attempts a {\tt push} for a scaffold entry that already contains a {\tt push}; intuitively this indicates the algorithm has returned to a previously visited position in the scaffold and is about to push again, a push that will result in returning to the entry yet again, ad infinitum.  Our chart-parsing algorithm initializes each scaffold entry $a_{in}$ with the value \texttt{pending}. The algorithm starts with entry $a_{00}$, and proceeds in a recursively top-down manner, updating each entry $a_{in}$.


\subsection{Example: Matching Parentheses}

Consider the language of matching parentheses where we want the parser to accept
the longest prefix of the input where parentheses have been matched
A simple PEG for this can be written as
\[ n_{\texttt{start}} \prodnInternal \,\texttt{"}\,\texttt{(}\,\texttt{"}\, n_{\texttt{start}} \,\texttt{"}\,\texttt{)}\,\texttt{"}\, n_{\texttt{start}} / \epsilon, \]
however this is not in normal form.  We can rewrite it into a normal form with seven productions, each of which is exactly one of the forms shown in Table~\ref{table:NormalForm}.  These productions are listed in the leftmost column of the scaffold in Figure \ref{scaffold:matchingParenthesis}.  Because, once in normal form, each non-terminal corresponds to exactly one production, these will also be the rows of the scaffold.  In Figure~\ref{scaffold:matchingParenthesis}, we compute matching parenthesis on the input "\texttt{(()())()}".

\begin{figure}[h!] %
\[
\begin{array}{|l||c|c|c|c|c|c|c|c|c|}\hline
& \texttt{(} & \texttt{(} & \texttt{)} & \texttt{(} & \texttt{)} & \texttt{)} & \texttt{(} & \texttt{)} & \epsilon \\ \hline \hline
n_{0} \prodnInternal n_{1} / n_{6} & g_{8}\tikzmark{a00} & g_{4}\tikzmark{a01} & g_{0}\tikzmark{a02} & g_{2}\tikzmark{a03} & g_{0}\tikzmark{a04} & g_{0}\tikzmark{a05} & g_{2}\tikzmark{a06} & g_{0}\tikzmark{a07} & g_{0}\tikzmark{a08} \\ \hline
n_{1} \prodnInternal n_{2} \circ n_{3} & g_{8}\tikzmark{a10} & g_{4}\tikzmark{a11} & f\tikzmark{a12} & g_{2}\tikzmark{a13} & f\tikzmark{a14} & f\tikzmark{a15} & g_{2}\tikzmark{a16} & f\tikzmark{a17} & f\tikzmark{a18} \\ \hline
n_{2} \prodnInternal \texttt{(} & g_{1}\tikzmark{a20} & g_{1}\tikzmark{a21} & f\tikzmark{a22} & g_{1}\tikzmark{a23} & f\tikzmark{a24} & f\tikzmark{a25} & g_{1}\tikzmark{a26} & f\tikzmark{a27} & f\tikzmark{a28} \\ \hline
n_{3} \prodnInternal n_{0} \circ n_{4} & & g_{7}\tikzmark{a31} & g_{3}\tikzmark{a32} & & g_{1}\tikzmark{a34} & & & g_{1}\tikzmark{a37} & \\ \hline
n_{4} \prodnInternal n_{5} \circ n_{0} & & & g_{3}\tikzmark{a42} & & g_{1}\tikzmark{a44} & g_{3}\tikzmark{a45} & & g_{1}\tikzmark{a47} & \\ \hline
n_{5} \prodnInternal \texttt{)} & & & g_{1}\tikzmark{a52} & & g_{1}\tikzmark{a54} & g_{1}\tikzmark{a55} & & g_{1}\tikzmark{a57} & \\ \hline
n_{6} \prodnInternal \epsilon & & & g_{0}\tikzmark{a62} & & g_{0}\tikzmark{a64} & g_{0}\tikzmark{a65} & & g_{0}\tikzmark{a67} & g_{0}\tikzmark{a68} \\ \hline
\end{array}
\]
\begin{tikzpicture}[overlay, remember picture, yshift=.25\baselineskip, shorten >=.5pt, shorten <=.5pt]
    \draw [->, blue] ({pic cs:a00}) [bend left] to ({pic cs:a10});
    \draw [->, blue] ({pic cs:a10}) [bend left] to ({pic cs:a20});
    \draw [->, blue] ({pic cs:a10}) [bend left=5] to ({pic cs:a31});
    \draw [->, blue] ({pic cs:a31}) [bend right] to ({pic cs:a01});
    \draw [->, blue] ({pic cs:a01}) [bend left] to ({pic cs:a11});
    \draw [->, blue] ({pic cs:a11}) [bend left] to ({pic cs:a21});
    \draw [->, blue] ({pic cs:a11}) [bend left=5] to ({pic cs:a32});
    \draw [->, blue] ({pic cs:a32}) [bend right] to ({pic cs:a02});
    \draw [->, red] ({pic cs:a02}) [bend left] to ({pic cs:a12});
    \draw [->, red] ({pic cs:a12}) [bend left] to ({pic cs:a22});
    \draw [->, blue] ({pic cs:a02}) [bend left=12] to ({pic cs:a62});
    \draw [->, blue] ({pic cs:a32}) [bend left] to ({pic cs:a42});
    \draw [->, blue] ({pic cs:a42}) [bend left] to ({pic cs:a52});
    \draw [->, blue] ({pic cs:a42}) [bend right=50] to ({pic cs:a03});
    \draw [->, blue] ({pic cs:a03}) [bend left] to ({pic cs:a13});
    \draw [->, blue] ({pic cs:a13}) [bend left] to ({pic cs:a23});
    \draw [->, blue] ({pic cs:a13}) [bend left=5] to ({pic cs:a34});
    \draw [->, blue] ({pic cs:a34}) [bend right] to ({pic cs:a04});
    \draw [->, red] ({pic cs:a04}) [bend left] to ({pic cs:a14});
    \draw [->, red] ({pic cs:a14}) [bend left] to ({pic cs:a24});
    \draw [->, blue] ({pic cs:a04}) [bend left=12] to ({pic cs:a64});
    \draw [->, blue] ({pic cs:a34}) [bend left] to ({pic cs:a44});
    \draw [->, blue] ({pic cs:a44}) [bend left] to ({pic cs:a54});
    \draw [->, blue] ({pic cs:a44}) [bend right=50] to ({pic cs:a05});
    \draw [->, red] ({pic cs:a05}) [bend left] to ({pic cs:a15});
    \draw [->, red] ({pic cs:a15}) [bend left] to ({pic cs:a25});
    \draw [->, blue] ({pic cs:a05}) [bend left=12] to ({pic cs:a65});
    \draw [->, blue] ({pic cs:a31}) [bend left=13] to ({pic cs:a45});
    \draw [->, blue] ({pic cs:a45}) [bend left] to ({pic cs:a55});
    \draw [->, blue] ({pic cs:a45}) [bend right=50] to ({pic cs:a06});
    \draw [->, blue] ({pic cs:a06}) [bend left] to ({pic cs:a16});
    \draw [->, blue] ({pic cs:a16}) [bend left] to ({pic cs:a26});
    \draw [->, blue] ({pic cs:a16}) [bend left=5] to ({pic cs:a37});
    \draw [->, blue] ({pic cs:a37}) [bend right] to ({pic cs:a07});
    \draw [->, red] ({pic cs:a07}) [bend left] to ({pic cs:a17});
    \draw [->, red] ({pic cs:a17}) [bend left] to ({pic cs:a27});
    \draw [->, blue] ({pic cs:a07}) [bend left=12] to ({pic cs:a67});
    \draw [->, blue] ({pic cs:a37}) [bend left] to ({pic cs:a47});
    \draw [->, blue] ({pic cs:a47}) [bend left] to ({pic cs:a57});
    \draw [->, blue] ({pic cs:a47}) [bend right=50] to ({pic cs:a08});
    \draw [->, red] ({pic cs:a08}) [bend left] to ({pic cs:a18});
    \draw [->, red] ({pic cs:a18}) [bend left] to ({pic cs:a28});
    \draw [->, blue] ({pic cs:a08}) [bend left=12] to ({pic cs:a68});
 \end{tikzpicture}
 \vspace*{-4mm}
 \caption{\small Matching Parenthesis on "\texttt{(()())()}".  This shows the contents of the scaffold after the parsing process has completed.  Blue arrows show the successful parse tree, and red arrows show the proof of a failed parse.  Scaffold entries marked $g_{i}$ indicate an entry of \emph{good} with a span of $i$, and entries marked $f$ indicate a fail.}
 \label{scaffold:matchingParenthesis}
 \end{figure}


\subsection{Example: Power of Two}

If we now want to accept any input stream that has is a positive power of two length or is empty (the length of the input is $0$ or one of $\lbrace 2, 4, 8, 16, \, \ldots \rbrace$).  The grammar 
\[ n_{\texttt{start}} \prodnInternal n_{\texttt{pow}} !\,.\, \hspace*{1cm} n_{\texttt{pow}} \prodnInternal \,.\,n_{\texttt{pow}}\,.\, / \,.\,(\&n_{\texttt{pow}})\,.\, / \epsilon \]
where "$.$" is any character, specifies this language.  Figure \ref{scaffold:powerOfTwo} shows this grammar running on the input "\texttt{parsed}".

\begin{figure}[h!] %tbp
\[
\begin{array}{|l||c|c|c|c|c|c|c|}\hline
& \texttt{p} & \texttt{a} & \texttt{r} & \texttt{s} & \texttt{e} & \texttt{d} &  \epsilon \\ \hline \hline
n_{0} \prodnInternal n_{1} \circ n_{10} & f\tikzmark{b00}~ & & & & & & \\ \hline
n_{10} \prodnInternal ! n_{3} & & & & & f~\tikzmark{ba4}~ & & \\ \hline
n_{8} \prodnInternal \& n_{1} & & & g_{0}~\tikzmark{b82}~ & & g_{0}~\tikzmark{b84}~ & & g_{0}~~\tikzmark{b86}~ \\ \hline
n_{1} \prodnInternal n_{2} / n_{5} & g_{4}\tikzmark{b10}~ & g_{2}~\tikzmark{b11}~ &g_{4}~\tikzmark{b12}~  & g_{2}~\tikzmark{b13}~ & g_{2}~\tikzmark{b14}~ & g_{0}~\tikzmark{b15}~ & g_{0}~~\tikzmark{b16}~ \\ \hline
n_{4} \prodnInternal n_{1} \circ n_{3} & & g_{3}~\tikzmark{b41}~  & f~\tikzmark{b42}~ & g_{3}~\tikzmark{b43}~ & f~\tikzmark{b44}~ & g_{1}~\tikzmark{b45}~ & f~~\tikzmark{b46}~ \\ \hline
n_{2} \prodnInternal n_{3} \circ n_{4} & g_{4}\tikzmark{b20}~ & f~\tikzmark{b21}~ & g_{4}~\tikzmark{b22}~ & f~\tikzmark{b23}~ & g_{2}~\tikzmark{b24}~ & f~\tikzmark{b25}~ & f~~\tikzmark{b26}~ \\ \hline
n_{3} \prodnInternal \texttt{any($true$)} & g_{1}\tikzmark{b30}~ & g_{1}~~\tikzmark{b31}~ & g_{1}~~\tikzmark{b32}~ & g_{1}~~\tikzmark{b33}~ & g_{1}~~\tikzmark{b34}~ & g_{1}~~\tikzmark{b35}~ & f~~\tikzmark{b36}~ \\ \hline
n_{5} \prodnInternal n_{6} / n_{9} & & g_{2}~\tikzmark{b51}~ & & g_{2}~\tikzmark{b53}~ & & g_{0}~\tikzmark{b55}~ & g_{0}~~\tikzmark{b56}~ \\ \hline
n_{9} \prodnInternal \epsilon & & & & & & g_{0}~\tikzmark{b95}~ & g_{0}~~\tikzmark{b96}~ \\ \hline
n_{6} \prodnInternal n_{3} \circ n_{7} & & g_{2}~\tikzmark{b61}~ & & g_{2}~\tikzmark{b63}~ & & f~\tikzmark{b65}~ & f~~\tikzmark{b66}~ \\ \hline
n_{7} \prodnInternal n_{8} \circ n_{3} & & & g_{1}~\tikzmark{b72}~ & & g_{1}~\tikzmark{b74}~ & & f~~\tikzmark{b76}~  \\ \hline
\end{array}
\]
\begin{tikzpicture}[overlay, remember picture, yshift=.25\baselineskip, shorten >=.5pt, shorten <=.5pt]
    \draw [->, blue] ({pic cs:b00}) [bend left] to ({pic cs:b10});
    \draw [->, blue] ({pic cs:b10}) [bend left] to ({pic cs:b20});
    \draw [->, blue] ({pic cs:b20}) [bend left] to ({pic cs:b30});
    \draw [->, blue] ({pic cs:b20}) [bend right=18] to ({pic cs:b41});
    \draw [->, blue] ({pic cs:b41}) [bend left] to ({pic cs:b11});
    \draw [->, red] ({pic cs:b11}) [bend left] to ({pic cs:b21});
    \draw [->, blue] ({pic cs:b21}) [bend left] to ({pic cs:b31});
    \draw [->, red] ({pic cs:b21}) [bend right=18] to ({pic cs:b42});
    \draw [->, blue] ({pic cs:b42}) [bend left] to ({pic cs:b12});
    \draw [->, blue] ({pic cs:b12}) [bend left] to ({pic cs:b22});
    \draw [->, blue] ({pic cs:b22}) [bend left] to ({pic cs:b32});
    \draw [->, blue] ({pic cs:b22}) [bend right=18] to ({pic cs:b43});
    \draw [->, blue] ({pic cs:b43}) [bend left] to ({pic cs:b13});
    \draw [->, red] ({pic cs:b13}) [bend left] to ({pic cs:b23});
    \draw [->, blue] ({pic cs:b23}) [bend left] to ({pic cs:b33});
    \draw [->, red] ({pic cs:b23}) [bend right=18] to ({pic cs:b44});
    \draw [->, blue] ({pic cs:b44}) [bend left] to ({pic cs:b14});
    \draw [->, blue] ({pic cs:b14}) [bend left] to ({pic cs:b24});
    \draw [->, blue] ({pic cs:b24}) [bend left] to ({pic cs:b34});
    \draw [->, blue] ({pic cs:b24}) [bend right=18] to ({pic cs:b45});
    \draw [->, blue] ({pic cs:b45}) [bend left] to ({pic cs:b15});
    \draw [->, red] ({pic cs:b15}) [bend left] to ({pic cs:b25});
    \draw [->, blue] ({pic cs:b25}) [bend left] to ({pic cs:b35});
    \draw [->, red] ({pic cs:b25}) [bend right=18] to ({pic cs:b46});
    \draw [->, blue] ({pic cs:b46}) [bend left] to ({pic cs:b16});
    \draw [->, red] ({pic cs:b16}) [bend left] to ({pic cs:b26});
    \draw [->, red] ({pic cs:b26}) [bend left] to ({pic cs:b36});
    \draw [->, blue] ({pic cs:b16}) [bend left=25] to ({pic cs:b56});
    \draw [->, red] ({pic cs:b56}) [bend left] to ({pic cs:b66});
    \draw [->, red] ({pic cs:b66}) [bend left=16] to ({pic cs:b36});
    \draw [->, blue] ({pic cs:b56}) [bend left=22] to ({pic cs:b96});
    \draw [->, red] ({pic cs:b46}) [bend left] to ({pic cs:b36});
    \draw [->, blue] ({pic cs:b15}) [bend left=25] to ({pic cs:b55});
    \draw [->, red] ({pic cs:b55}) [bend left] to ({pic cs:b65});
    \draw [->, blue] ({pic cs:b65}) [bend left=16] to ({pic cs:b35});
    \draw [->, red] ({pic cs:b65}) [bend left=5] to ({pic cs:b76});
    \draw [->, blue] ({pic cs:b76}) [bend right=18] to ({pic cs:b86});
    \draw [->, blue] ({pic cs:b86}) [bend left] to ({pic cs:b16});
    \draw [->, red] ({pic cs:b76}) [bend left=12] to ({pic cs:b36});
    \draw [->, blue] ({pic cs:b55}) [bend left=22] to ({pic cs:b95});
    \draw [->, blue] ({pic cs:b45}) [bend left] to ({pic cs:b35});
    \draw [->, red] ({pic cs:b44}) [bend left=13] to ({pic cs:b36});
     \draw [->, blue] ({pic cs:b43}) [bend left] to ({pic cs:b33});
    \draw [->, blue] ({pic cs:b13}) [bend left=25] to ({pic cs:b53});
    \draw [->, blue] ({pic cs:b53}) [bend left] to ({pic cs:b63});
    \draw [->, blue] ({pic cs:b63}) [bend left=16] to ({pic cs:b33});
    \draw [->, blue] ({pic cs:b63}) [bend left=5] to ({pic cs:b74});
    \draw [->, blue] ({pic cs:b74}) [bend right=18] to ({pic cs:b84});
    \draw [->, blue] ({pic cs:b84}) [bend left] to ({pic cs:b14});
    \draw [->, blue] ({pic cs:b74}) [bend left=12] to ({pic cs:b34});
    \draw [->, red] ({pic cs:b42}) [bend left=7] to ({pic cs:b36});
    \draw [->, blue] ({pic cs:b11}) [bend left=25] to ({pic cs:b51});
    \draw [->, blue] ({pic cs:b51}) [bend left] to ({pic cs:b61});
    \draw [->, blue] ({pic cs:b61}) [bend left=16] to ({pic cs:b31});
    \draw [->, blue] ({pic cs:b61}) [bend left=5] to ({pic cs:b72});
    \draw [->, blue] ({pic cs:b72}) [bend right=18] to ({pic cs:b82});
    \draw [->, blue] ({pic cs:b82}) [bend left] to ({pic cs:b12});
    \draw [->, blue] ({pic cs:b72}) [bend left=12] to ({pic cs:b32});
    \draw [->, blue] ({pic cs:b41}) [bend left] to ({pic cs:b31});
    \draw [->, red] ({pic cs:b00}) [bend right=10] to ({pic cs:ba4});
    \draw [->, blue] ({pic cs:ba4}) [bend left] to ({pic cs:b34});
    
 \end{tikzpicture}
 \vspace*{-4mm}
 \caption{\small Power of two length on "\texttt{parsed}".  The rows have been rearranged to make it easier to follow the arrows.  Note in the $n_{1}$ row there reason to believe this grammar would accept inputs of length $2$, $4$, and $0$.  This is because the $n_{10}$ just checks that $n_{1}$ consumed all the input, so if the string started with the {\tt r} (meaning it had length $4$), then $n_{0}$ would correctly accept and consume all $4$ characters.}
 \label{scaffold:powerOfTwo}
 \end{figure}



\subsection{Example: Loop}
We can also consider a grammar which is not well formed.  The following will accept strings that are a sequence of \emph{blocks} ending with a {\tt \#}.  Each block is either {\tt a}, {\tt b}, or {\tt ab} (but not {\tt ba}, which would instead be parsed as two separate blocks).  The grammar
\[ n_{\texttt{start}} \prodnInternal n_{\texttt{block}} ( \texttt{\#} / n_{\texttt{start}} ) \hspace*{1cm} n_{\texttt{block}} \prodnInternal !\texttt{c} (\texttt{a} / \epsilon) (\texttt{b} / \epsilon) \]
correctly parses all correct instances of this.  Further it will successfully reject any string that contains the character {\tt c}.  This means on the language $\lbrace \texttt{a}, \texttt{b}, \texttt{c}, \texttt{\#} \rbrace$ it will behave properly, however if it encounters something other than {\tt a}, {\tt b}, {\tt c}, or {\tt \#}  it will loop.  Figure \ref{scaffold:loop} shows this grammar mid-computation on the input {\tt abbda\#}.

\begin{figure}[h!] %tbp
\[
\begin{array}{|l||c|c|c|c|c|c|c|c|}\hline
& \texttt{a} & \texttt{b} & \texttt{b} & \texttt{d} & \texttt{a} & \texttt{\#} &  \epsilon \\ \hline \hline
n_{0} \prodnInternal n_{1} \circ n_{10} & p_{2,10}\tikzmark{c00} & & p_{3,10}\tikzmark{c02} & l\tikzmark{c03} & & & \\ \hline
n_{1} \prodnInternal n_{2} \circ n_{4} & g_{2}\tikzmark{c10} & & g_{1}\tikzmark{c12} & g_{0}\tikzmark{c13} & & & \\ \hline
n_{2} \prodnInternal !n_{3} & g_{0}\tikzmark{c20} & & g_{0}\tikzmark{c22} & g_{0}\tikzmark{c23} & & & \\ \hline
n_{3} \prodnInternal \texttt{c} & f\tikzmark{c30} & & f\tikzmark{c32} & f\tikzmark{c33} & & & \\ \hline
n_{4} \prodnInternal n_{5} \circ n_{8} & g_{2}\tikzmark{c40} & & g_{1}\tikzmark{c42} & g_{0}\tikzmark{c43} & & & \\ \hline
n_{5} \prodnInternal n_{6} / n_{7} & g_{1}\tikzmark{c50} & & g_{0}\tikzmark{c52} & g_{0}\tikzmark{c53} & & & \\ \hline
n_{6} \prodnInternal \texttt{a} & g_{1}\tikzmark{c60} & & f\tikzmark{c62} & f\tikzmark{c63} & & & \\ \hline
n_{7} \prodnInternal \epsilon & & & g_{0}\tikzmark{c72} & g_{0}\tikzmark{c73} & & & \\ \hline
n_{8} \prodnInternal n_{9} / n_{7} & & g_{1}\tikzmark{c81} & g_{1}\tikzmark{c82} & g_{0}\tikzmark{c83} & & & \\ \hline
n_{9} \prodnInternal \texttt{b} & & g_{1}\tikzmark{c91} & g_{1}\tikzmark{c92} & f\tikzmark{c93} & & & \\ \hline
n_{10} \prodnInternal n_{11} / n_{0} & & & p_{2,0}\tikzmark{ca2} & p_{3,0}\tikzmark{ca3} & & & \\ \hline
n_{11} \prodnInternal \texttt{\#} & & & f\tikzmark{cb2} & f\tikzmark{cb3} & & & \\ \hline
\end{array}
\]
\begin{tikzpicture}[overlay, remember picture, yshift=.25\baselineskip, shorten >=.5pt, shorten <=.5pt]
    \draw [->, blue] ({pic cs:c00}) [bend left] to ({pic cs:c10});
    \draw [->, blue] ({pic cs:c10}) [bend left] to ({pic cs:c20});
    \draw [->, red] ({pic cs:c20}) [bend left] to ({pic cs:c30});
    \draw [->, blue] ({pic cs:c10}) [bend left] to ({pic cs:c40});
    \draw [->, blue] ({pic cs:c40}) [bend left] to ({pic cs:c50});
    \draw [->, blue] ({pic cs:c50}) [bend left] to ({pic cs:c60});
    \draw [->, blue] ({pic cs:c40}) [bend right=5] to ({pic cs:c81});
    \draw [->, blue] ({pic cs:c81}) [bend left] to ({pic cs:c91});
    \draw [->, orange] ({pic cs:c00}) [bend right=15] to ({pic cs:ca2});
    \draw [->, red] ({pic cs:ca2}) [bend left] to ({pic cs:cb2});
    \draw [->, orange] ({pic cs:ca2}) [bend right=7] to ({pic cs:c02});
    \draw [->, blue] ({pic cs:c02}) [bend left] to ({pic cs:c12});
    \draw [->, blue] ({pic cs:c12}) [bend left] to ({pic cs:c22});
    \draw [->, red] ({pic cs:c22}) [bend left] to ({pic cs:c32});
    \draw [->, blue] ({pic cs:c12}) [bend left] to ({pic cs:c42});
    \draw [->, blue] ({pic cs:c42}) [bend left] to ({pic cs:c52});
    \draw [->, red] ({pic cs:c52}) [bend left] to ({pic cs:c62});
    \draw [->, blue] ({pic cs:c52}) [bend left] to ({pic cs:c72});
    \draw [->, blue] ({pic cs:c42}) [bend left] to ({pic cs:c82});
    \draw [->, blue] ({pic cs:c82}) [bend left] to ({pic cs:c92});
    \draw [->, orange] ({pic cs:c02}) [bend right=10] to ({pic cs:ca3});
    \draw [->, red] ({pic cs:ca3}) [bend left] to ({pic cs:cb3});
    \draw [->, orange] ({pic cs:ca3}) [bend right=25] to ({pic cs:c03});
    \draw [->, blue] ({pic cs:c03}) [bend left] to ({pic cs:c13});
    \draw [->, blue] ({pic cs:c13}) [bend left] to ({pic cs:c23});
    \draw [->, red] ({pic cs:c23}) [bend left] to ({pic cs:c33});
    \draw [->, blue] ({pic cs:c13}) [bend left] to ({pic cs:c43});
    \draw [->, blue] ({pic cs:c43}) [bend left] to ({pic cs:c53});
    \draw [->, red] ({pic cs:c53}) [bend left] to ({pic cs:c63});
    \draw [->, blue] ({pic cs:c53}) [bend left] to ({pic cs:c73});
    \draw [->, blue] ({pic cs:c43}) [bend left] to ({pic cs:c83});
    \draw [->, red] ({pic cs:c83}) [bend left] to ({pic cs:c93});
    \draw [->, blue] ({pic cs:c83}) [bend right] to ({pic cs:c73});
    \draw [->, green] ({pic cs:c03}) [bend left=40] to ({pic cs:ca3});
    
 \end{tikzpicture}
 \vspace*{-4mm}
 \caption{\small Computation of the block grammar on the string "{\tt abbda\#}".  The scaffold is shown midway through its computation at the time it detected the loop. The entries in the form $p_{i,j}$ show {\tt push} entries with position $i$ and non-terminal $j$, and the $l$ entry shows a {\tt loop}.  The orange arrows between the {\tt push} entries is the embedded call stack, while the green arrow shows the relationship that first detected the loop.  If this scaffold were to continue running, the {\tt push} entries would become {\tt loop} entries as the program backed its way up the embedded stack where it will terminate and declare a loop.  Note if this {\tt d}, were a {\tt c} then it would have already rejected by this point, and if the {\tt d} were an {\tt a} or a {\tt b}, then it would eventually accept the whole input.}
 \label{scaffold:loop}
 \end{figure}








\section{Chart Parsing: Formalization and Proof}\label{sec:chartparsing}


We now present a formalization (\emph{available on request})
of the chart parsing algorithm in PVS.
This formalization exploits specific features of
the PVS dependently subtyped higher-order logic.  
We quickly summarize the basic features and 
explain more advanced  features of the PVS specification language
as needed to understand the formalization.  

\myparagraph{PVS in Brief.} The PVS specification language is based on strongly typed higher-order logic.  
A PVS specification is a collection of theories.  Each theory has
some type and individual parameters, and the body of the theory is a sequence
of declarations of types and constants.  A type can be one of the built-in
types; a (possibly dependent) tuple, record, or function type, a predicate
subtype of the form $\{x: T ~|~ p(x)\}$, or an algebraic or coalgebraic datatype.
The type \texttt{nat} of natural numbers is a predicate subtype of the
type \texttt{int} of integers, which in turn is a subtype \texttt{rat} of the
rational numbers, and the real number type \texttt{real}\@.  Dependent
types can be used, for example, to define the type of permutations over a
given initial segment of the natural numbers as a dependent record with a
\texttt{size} field over \texttt{nat}, and a permutation field that is an bijective
map from \texttt{below(size)}, which is defined as $\{\mathtt{i} : \mathtt{nat} | \mathtt{i} < \mathtt{size}\}$, to itself.  The PVS type system can thus express
the specification of a function using the type system.  
 The PVS typechecker generates proof obligations called type correctness conditions (TCCs) corresponding to subtype constraints and termination.  
These proof obligations can be discharged through automated proof strategies or interactively developed proofs.  The PVS theorem prover combines powerful back-end automation in the form of simplification, rewriting, and satisfiability solving with
user-defined proof strategies allowing proofs to be constructed at varying levels of automation, transparency, and user control.



% \begin{figure}[h!]
% 			{\lstinputlisting[language=PVS]{code_samples/pegtopdown.txt}}
% 			\caption{\vspace{-.3cm}The theory \texttt{pegtopdown} with the contents elided.}
% 			\label{pvs:pegtopdown}
% \end{figure}

\subsection{Formal Specification of a Chart Parser}
\myparagraph{Some Basic Types}
Our PVS development of the PEG chart parser takes place in a single theory
titled \texttt{pegtopdown}.
The first part of the theory defines a few basic types and a single
constant.  The type \texttt{byte} is a subtype of natural numbers from
$0$ to $255$\@.  The type \texttt{strings(len)} is a subtype
\texttt{string} type (which defined in the PVS prelude library as a
finite sequence) consisting of those character sequences that are of
length \texttt{len}\@.  The number of nonterminals
\texttt{num\_non\_terminals} is specified as $255$\@.\footnote{We could
  have left the number of nonterminals unbounded and it would have
  made very little difference to the proof and the resulting programs
  would still be executable.}  These bounds are specified in order to
make the proofs highlight some of the complexity introduced in
handling machine-representable integers. Also, we plan to use
the PVS2C code generator to produce efficient imperative code implementing
the parser.  We also declare a variable \texttt{len} which we use to represent the length
of the input string to be parsed.  The two typing judgements assert the
subtype relationships between \texttt{upto(len)} and \texttt{uint32},
and \texttt{non\_terminal} and \texttt{uint8}\@.  These yield proof
obligations that need to be proved, but the typechecker uses these
judgements without generating further proof obligations, for example,
when an expression of type \texttt{non\_terminal} is required to match
the expected type \texttt{uint32}\@.  
\begin{figure}[h!]
  {\lstinputlisting[language=PVS]{code_samples/byte.txt}}
  \vspace*{-4mm}
			\caption{Basic type definitions.}
			\label{pvs:byte}
\end{figure}


\myparagraph{Scaffold Entries.}
The entry type for the scaffold (Figure~\ref{pvs:ent})
is an algebraic datatype with
constructors \texttt{fail} (with accessor \texttt{dep} representing
the \emph{nesting depth} (or height) and recognizer \texttt{fail?}), \texttt{pending}
(with no accessors and recognizer \texttt{pending?}), \texttt{loop}
(with recognizer \texttt{loop?}), \texttt{good} (with accessors
\texttt{dep} and \texttt{span} and recognizer \texttt{good?}), and
\texttt{push} (with accessors \texttt{pos} and \texttt{nt}, and
recognizer \texttt{push?}).  The values of type \texttt{ent} are
entries in the scaffold.  For example, the entry \texttt{fail} at
position $i$ and $nonterminal$ $n$ represents a failed parse for
nonterminal $n$ at position $i$.

\begin{figure}[h!]
  {\lstinputlisting[language=PVS]{code_samples/ent.txt}}
  \vspace*{-4mm}
			\caption{The entry type for the scaffold.}
			\label{pvs:ent}
\end{figure}


\myparagraph{PEG expressions.}
The type of PEG expressions is defined in Figure~\ref{pvs:peg} by a datatype \texttt{peg}.  The
constructor \texttt{epsilon} represents the empty string,
\texttt{failure} always parses to a failure, \texttt{any(p)} matches
an input character satisfying the predicate \texttt{p},
\texttt{terminal(a)} matches the character \texttt{a}, the
concatenation operation $\mathtt{concat(n_1, n_2)}$ only allows nonterminals
as the sub-grammars, and similarly for order choice
\texttt{choice(A)}, the lookahead check \texttt{check(A)} (which
corresponds to the $\&$ operation), and the negation operation
\texttt{neg(A)}\@.

\begin{figure}[h!]
  {\lstinputlisting[language=PVS]{code_samples/peg.txt}}
  \vspace*{-4mm}
			\caption{The PEG syntax}
			\label{pvs:peg}
\end{figure}

Figure~\ref{pvs:grammar} defines the type \texttt{lang\_spec} of grammars, i.e., the set of productions, as 
 a map from the type \texttt{non\_terminal} to \texttt{peg}\@.  
\begin{figure}[h!]
  \lstinputlisting[language=PVS]{code_samples/grammar.txt}
    \vspace*{-4mm}
\caption{\small Language Specification Type}
\label{pvs:grammar}
\end{figure}


\myparagraph{Valid Scaffold Entries.}
The predicates defined in Figure~\ref{pvs:checkentry} capture stronger checks on the entries in the scaffold.
These checks are not built into the definition of \texttt{ent} since
they require the context of the length of the string and the position
of the entry within the scaffold.\footnote{One could define \texttt{ent} as a
  parameterized datatype, but this has the messy consequence that
   the individual entries in the scaffold would
be from different instantiations of the datatype. }
The predicate \texttt{good\_good\_entry?} checks that
 \texttt{i + j} is no greater than \texttt{len} 
for an entry of the form \texttt{good(d, i)} with span length \texttt{i}
at position \texttt{j} in the scaffold.  The predicate \texttt{good\_push\_entry?}
checks for any entry of the form \texttt{push(i, n)} in the scaffold
that $\mathtt{i} \leq \mathtt{len}$ (since the position where $\mathtt{i} = \mathtt{len}$ also has a scaffold entry) 
and $ \mathtt{n}\leq \mathtt{num\_non\_terminals}$ (since $\mathtt{n} = \mathtt{num\_non\_terminals}$ is used to mark the end of the stack).  A stronger predicate
\texttt{fine\_push\_entry?} checks for \texttt{push(i, n)} that $\mathtt{n} < \mathtt{num\_non\_terminals}$ indicating that it is a proper stack entry and not the
end of the stack.  We represent the stack such that if the
top of the stack is at the position $(i_0, n_0)$ of the scaffold $A$, then
either $n_0 = \mathtt{num\_non\_terminals}$ and the stack is empty
(and $i_0$ is irrelevant), or $A(i_0)(n_0) = \mathtt{push}(i_1, n_1)$, and
$(i_1, n_1)$ is the next element in the stack.  Later on, we introduce further conditions
to ensure that the stack is well-formed.  The predicate \texttt{nice\_entry?} checks that entries of the form \texttt{good(d, i)} and \texttt{push(i, n)} satisfy \texttt{good\_good\_entry?} and \texttt{good\_push\_entry?}, respectively.

The predicate \texttt{loop\_or\_push?} checks that the entry is of the form
\texttt{loop} or \texttt{push(i, n)}\@.  This predicate is needed 
in order capture an invariant for the parsing algorithm.

\begin{figure}[h!]
  \lstinputlisting[language=PVS]{code_samples/checkentry.txt}
    \vspace*{-4mm}
			\caption{Predicates for checking the scaffold entries.}
			\label{pvs:checkentry}
\end{figure}


\myparagraph{The Scaffold Type.}
With the types and predicates defined above, Figure~\ref{pvs:scaffold}
defines the type of a
scaffold as an array over \texttt{upto(len)} of arrays over \texttt{non\_terminal}
where each entry is meets the \texttt{nice\_entry?} condition\@.  
\begin{figure}[h!]
  \lstinputlisting[language=PVS]{code_samples/scaffold.txt}
    \vspace*{-4mm}
			\caption{The scaffold type}
			\label{pvs:scaffold}
\end{figure}


\myparagraph{Stack Validity.}
Next, we need to characterize a well-formed stack.  In an earlier
attempt, we used a stack datatype that we could ensure was
well-formed.   We first tried to economize by not placing any
constraints other than identifying stack entries $(\mathtt{i, n})$ in the
scaffold $A$ as those where $A(i)(n)$ was of the form \texttt{push(i',
  n')}\@.  Unfortunately, the interactive proofs quickly made it clear that if a stack entry points
to a non-stack entry, we terminate in a state where some scaffold
entries that are marked as being on the stack have not yet been
processed.  We could insist that every entry of the form
\texttt{push(i', n')} in the scaffold $A$, it should be the case that
$A(i')(n')$ is also of the form \texttt{push(i'', n'')}.  However,
this condition can hold when there are cycles and it would be negated
when any entry in the cycle is updated to a \texttt{fail},
\texttt{loop}, or \texttt{good} value.  Occasionally, these
experiments do succeed, and either way, we learn something from both
failure and success with a modest amount of manual effort.


\myparagraph{Stack Successor.}
  The predicate \texttt{successor} defined in Figure~\ref{pvs:successor}
checks if \texttt{entry2} is a successor of \texttt{entry1} for two stack entries \texttt{entry1} and \texttt{entry2}\@.
Given a grammar \texttt{G} and a scaffold \texttt{A}, the \texttt{successor} predicate checks that \texttt{entry2} corresponds to a sub-query of the query \texttt{entry1}\@.  For example, if \texttt{entry1} is of the form \texttt{push(p1, nt1)}
and \texttt{entry2} is of the form \texttt{push(p2, nt2)}, and \texttt{G(nt1)}
is of the form \texttt{concat(n1, n2)}, then either \texttt{entry2} corresponds to the \texttt{n1} successor in which case $\mathtt{p1} = \mathtt{p2}$, or \texttt{A(p1)(n1)} has a \texttt{good} entry, and \texttt{entry2} is the \texttt{n2} successor of \texttt{entry1} so that $\mathtt{p2} = \mathtt{p1 = span(A(p1)(n1))}$ and $\mathtt{nt2} = \mathtt{n2}$\@.  The other cases of the definition are similar.  
\begin{figure}[h!]
  \lstinputlisting[language=PVS]{code_samples/successor.txt}
    \vspace*{-4mm}
			\caption{The successor relation on stack entries}
			\label{pvs:successor}
\end{figure}

\myparagraph{Well-formed Stack.}
We use the \texttt{successor} relation to characterize a good stack within
a scaffold.  The recursive predicate \texttt{good\_stack?} take a grammar \texttt{G}, a scaffold \texttt{A}, and an entry \texttt{stack} that is required to
satisfy the \texttt{good\_push\_entry?} predicate.  It also takes a \texttt{depth}
parameter to ensure that the recursion terminates.  When \texttt{stack} is empty,
i.e., $\mathtt{nt(stack)} \geq \mathtt{num\_non\_terminals}$,
the \texttt{depth} must be $0$.  Otherwise, and in this case
$0\leq \mathtt{nt(stack)} < \mathtt{num\_non\_terminals}$, we can look up
$\mathtt{A(pos(stack))(nt(stack))}$ and bind it to \texttt{entry}\@.  We check that
\texttt{entry} satisfies the \texttt{good\_push\_entry?} predicate,
the successor relation holds between \texttt{entry} and \texttt{stack}
when \texttt{entry} is nonempty, the \texttt{depth} is positive, and
\texttt{entry} also recursively satisfies the \texttt{good\_stack?} predicate.
The well-founded termination measure is given by the \texttt{depth} parameter
which decreases by one in the recursive invocation.  
\begin{figure}[h!]
  \lstinputlisting[language=PVS]{code_samples/goodstack.txt}
    \vspace*{-4mm}
			\caption{Checking that the stack is well-formed. }
			\label{pvs:goodstack}
\end{figure}

\myparagraph{Stack Membership.}
For loop detection, we need to be able to check membership in the stack.
This is done by the \texttt{mem\_stack?} predicate whose definition looks
similar to that of \texttt{good\_stack?}\@.  The \texttt{mem\_stack?} predicate is only employed in the specification and is not used in defining the parsing
state machine.  
                      
\begin{figure}[h!]
  \lstinputlisting[language=PVS]{code_samples/memstack.txt}
    \vspace*{-4mm}
			\caption{Checking membership in the stack. }
			\label{pvs:memcheck}
\end{figure}

The \texttt{mem\_stack?} predicate is used to define the predicate \texttt{fine\_stack?} which checks that there are no duplicate elements in the stack.   It turns out there can be no duplicate elements in a stack if it satisfies the \texttt{good\_stack?} predicate since it can be shown that each element of the stack occurs at a unique depth.  However, maintaining the \texttt{fine\_stack?} condition as defined is simpler than proving that 
it is already entailed by \texttt{good\_stack?}.

\begin{figure}[h!]
  \lstinputlisting[language=PVS]{code_samples/finestack.txt}
    \vspace*{-4mm}
			\caption{A stack with no repeating elements }
			\label{pvs:finestack}
\end{figure}

\myparagraph{Scaffold Consistency.}
Having dealt with the stack portion of the scaffold, we are now ready to characterize the validity conditions on the other non-\texttt{pending} entries in the scaffold.
The first of these, \texttt{loop\_ready?}, defined in Figure~\ref{pvs:loopready}, checks if the cell at position \texttt{i} and nonterminal \texttt{n} is a candidate that could be marked as a loop.
Essentially, a cell is \texttt{loop\_ready?} if it has a successor that
is either marked as a \texttt{loop} or is on the stack, i.e., is a \texttt{push} entry.   
\begin{figure}[h!]
  \lstinputlisting[language=PVS]{code_samples/loopready.txt}
    \vspace*{-4mm}
			\caption{Condition under which a cell can be marked as a loop}
			\label{pvs:loopready}
                      \end{figure}

The \texttt{good\_fail?} predicate checks the parse should fail at position \texttt{i} of the input string for nonterminal \texttt{n}\@.  For example, The parse could fail because \texttt{G(n)} is of the form \texttt{any(p)} and \texttt{p} fails on the character at position \texttt{i} of the input \texttt{s}\@.  In another example, if \texttt{G(n)} has the form \texttt{concat(n1, n2)}, then the parse could fail at \texttt{i} for \texttt{n1}, or at \texttt{i + span(A(i)(n1))} for \texttt{n2}\@.  The \texttt{fail} entry also tracks the nesting depth of the failure.  We omit the definition of
                      \texttt{good\_fail?} since it is similar to that of
                      \texttt{good\_good?} below. 
                      
% \begin{figure}[h!]
%   \lstinputlisting[language=PVS]{code_samples/goodfail.txt}
% 			\caption{Condition under which a cell can be marked as \texttt{fail}}
% 			\label{pvs:goodfail}
% \end{figure}

                      
% The next validity check on a scaffold entry is for cells marked as \texttt{loop}
% as defined by the \texttt{good\_loop?} predicate.  This is similar to the
% \texttt{loop\_ready?} predicate, but with the weaker \texttt{loop\_or\_push?}  test replaced by \texttt{loop?} since \texttt{good\_loop?} is meant to characterize
% a completed chart. 
                      
% \begin{figure}[h!]
% 			\lstinputlisting[language=PVS]{code_samples/goodloop.txt}
% 			\caption{Condition under which a cell represents a fully detected \texttt{loop}}
% 			\label{pvs:goodloop}
%                       \end{figure}

                      The \texttt{good\_good?} predicate checks that \texttt{A(i)(n)} can be marked
                      with \texttt{good(d, sp)}\@.  The case analysis is on the grammar given by \texttt{G(n)}\@.  For example, if \texttt{G(n)} is \texttt{epsilon}, then the entry must be \texttt{good(0, 0)}\@.  If \texttt{G(n)} is \texttt{choice(n1, n2)}, then either \texttt{A(i, n1)} is \texttt{good(d1, sp)}, with $\mathtt{d} = 1 + \mathtt{d1 + 1}$, or
$\mathtt{A(i)(n1)} = \mathtt{fail(d1)}$ and $\mathtt{A(i)(n2)} = \mathtt{good(d2, sp)}$, with $\mathtt{d} = \mathtt{1 + max(d1, d2)}$\@.  
                      



The \texttt{good\_good?} predicate defined in
Figure~\ref{pvs:goodgood} checks that a cell $\mathtt{i, n})$ can be
marked with \texttt{good(d, sp)}\@.  The case analysis is on the
grammar given by \texttt{G(n)}\@.  For example, if \texttt{G(n)} is
\texttt{epsilon}, then the entry must be \texttt{good(0, 0)}\@.  If
\texttt{G(n)} is \texttt{choice(n1, n2)}, then either \texttt{A(i,
  n1)} is \texttt{good(d1, sp1)}, and $\mathtt{sp} = \mathtt{sp1}$ and
$\mathtt{d} = 1 + \mathtt{d1 + 1}$, or
$\mathtt{A(i)(n1)} = \mathtt{fail(d1)}$ and
$\mathtt{A(i)(n2)} = \mathtt{good(d2, sp2)}$,
$\mathtt{sp} = \mathtt{sp2}$,
$\mathtt{d} = \mathtt{1 = max(d1, d2)}$\@.
The need for recording the nesting depth within the scaffold entry was
discovered during the proof.  Without it, we run into a problem where
a scaffold could mark a cell with a self-loop with a \texttt{good} entry. 
This would satisfy the \texttt{good\_good?} condition, but there would be no
way to construct a well-formed parse tree from such a loop.  

\begin{figure}[h!]
  \lstinputlisting[language=PVS]{code_samples/goodgood.txt}
    \vspace*{-4mm}
			\caption{Condition allowing a cell to be marked as \texttt{good}}
			\label{pvs:goodgood}
\end{figure}

The checks defined by \texttt{good\_fail?}, \texttt{loop\_ready?}, and \texttt{good\_good?} are used to define \texttt{good\_entry?} in Figure~\ref{pvs:goodentry}
as a check on any entry \texttt{u}
at cell $(\mathtt{i, n})$ in the scaffold.  

\begin{figure}[h!]
  \lstinputlisting[language=PVS]{code_samples/goodentry.txt}
    \vspace*{-4mm}
\caption{Condition under which a cell can be marked as \texttt{good},
  \texttt{loop}, or \texttt{fail}}
\label{pvs:goodentry}
\end{figure}

With the definition of \texttt{good\_entry?}, we can start to characterize
a well-formed scaffold data structure as an invariant that can be preserved
during parsing leading to a post-condition that establishes the correctness
of the parse.  On the one hand, we can simply assert that each entry \texttt{A(i)(n)} at position \texttt{i} for nonterminal \texttt{n}
satisfies the predicate
$$\mathtt{good\_entry?(len, G, s)(A, i, n, A(i)(n))}.$$  However, one
thorny issue is that as we update the entries, we also have to establish that their
nesting depth is within the \texttt{uint64} subrange.  We need to maintain
an invariant bounding the nesting depth entries so that they can be shown to
remain within the range representable in \texttt{uint64}\@.  The nesting
depth can be bounded by the cardinality of the subset of entries that are
either \texttt{good} or \texttt{fail}, since these are the only entries that
represent nesting depth.

\myparagraph{Cardinalities.}
The operation \texttt{scafcount} defined in Figure~\ref{pvs:scafcount}
uses the summation operation from
the PVS NASAlib \texttt{reals} library to define the cardinality of a set (predicate)
over the scaffold entries.  

\begin{figure}[h!]
  \lstinputlisting[language=PVS]{code_samples/scafcount.txt}
    \vspace*{-4mm}
\caption{Computing cardinalities over the scaffold}
\label{pvs:scafcount}
\end{figure}

We can use the \texttt{scafcount} operation to define cardinalities
for some specific sets.  For example, \texttt{pushcount} defines the
cardinality of the set of \texttt{push} entries in the scaffold.
The \texttt{gfcount} operation defines the cardinality of the set
of \texttt{good} or \texttt{fail} entries in the scaffold.

\begin{figure}[h!]
  \lstinputlisting[language=PVS]{code_samples/gfcount.txt}
    \vspace*{-4mm}
\caption{Computing cardinalities over the scaffold of push entries, and good or fail entries}
\label{pvs:gfcount}
\end{figure}


\myparagraph{Scaffold Invariant.}
The predicate \texttt{good\_tscaffold?} defined in Figure~\ref{pvs:goodtscaffold}
captures the correctness of the scaffold
data structure (ignoring the stack) by asserting that each  entry
satisfies the \texttt{good\_entry?} predicate, and any nested depth field
in an entry is bounded by the cardinality of the set of \texttt{good} or \texttt{failed} nodes in the scaffold.  
\begin{figure}[h!]
  \lstinputlisting[language=PVS]{code_samples/goodtscaffold.txt}
    \vspace*{-4mm}
\caption{Main Scaffold Invariant}
\label{pvs:goodtscaffold}
\end{figure}

  We want to make sure that all the   \texttt{push} entries in the scaffold are part of stack.  For this to hold, the depth of the stack must be identical to the cardinality of the set of \texttt{push} entries in the scaffold as captured by the \texttt{good\_depth?} predicate defined in Figure~\ref{pvs:gooddepth}.

\begin{figure}[h!]
  \lstinputlisting[language=PVS]{code_samples/gooddepth.txt}
    \vspace*{-4mm}
\caption{Depth Invariant}
\label{pvs:gooddepth}
\end{figure}

How do we know that when the parser has completed its work and the finalized entry
for the root query has been entered?  By insisting, as the predicate
\texttt{good\_root?} does that \texttt{A(rootpos)(rootnt)}, for
the root position \texttt{rootpos} and root nonterminal \texttt{rootnt},
is never \texttt{pending}, we know that this entry in the scaffold is either on the stack or it is finalized.  Parsing terminates when the stack is empty so the
root entry must be a finalized entry.  

\begin{figure}[h!]
  \lstinputlisting[language=PVS]{code_samples/goodroot.txt}
    \vspace*{-4mm}
\caption{The root query must be on the stack if the stack is nonempty}
\label{pvs:goodroot}
\end{figure}

The final invariant on the scaffold data structure is defined in
\texttt{fine\_scaffold?} as a conjunction of \texttt{good\_tscaffold?}
and \texttt{good\_root?}\@.  
\begin{figure}[h!]
  \lstinputlisting[language=PVS]{code_samples/finescaffold.txt}
    \vspace*{-4mm}
\caption{A fine scaffold has all good entries and a good root}
\label{pvs:finescaffold}
\end{figure}

\subsection{Defining the Parser Interpreter}

\myparagraph{Parser State.}
We now have almost everything we need to characterize the state of our
state machine.  The type is specified as a dependent record with fields:
\begin{enumerate}
\item \texttt{scaf} for the scaffold which is typed to satisfy the predicate \texttt{fine\_scaffold?}.\footnote{If \texttt{p?} is a predicate on type \texttt{T}, then \texttt{(p?)} is short for \texttt{\{x : T | p?(x)\}}.}
\item \texttt{depth} for the stack depth which is typed to satisfy the predicate \texttt{good\_depth?}.
\item \texttt{stack} which is the entry corresponding to the scaffold location for
  the top of the stack, and typed to satisfy the \texttt{fine\_stack?} predicate.
\item \texttt{lflag}, a loop detection flag that is \texttt{TRUE} when a loop has been detected.  The type constraint on \texttt{lflag} is that the top of the stack must satisfy \texttt{loop\_ready?} when a loop has been detected, and there must be no \texttt{loop} entries in the scaffold, otherwise.  
\end{enumerate}
One important point about the dependent typing of the \texttt{state} record type
is that the field types are carefully staged so that each field can be typed 
in terms of the fields that precede it.  In particular, the scaffold can be updated independently.  
\begin{figure}[h!]
  \lstinputlisting[language=PVS]{code_samples/state.txt}
    \vspace*{-4mm}
\caption{Parser State Type}
\label{pvs:state}
\end{figure}

We can define a predicate \texttt{empty?} that checks the emptiness of a stack
entry, and use it to define a subtype \texttt{ne\_state} of the \texttt{state}
type separating those states with nonempty stacks.  The parser is always invoked with a nonempty stack containing the root query.  
\begin{figure}[h!]
  \lstinputlisting[language=PVS]{code_samples/empty.txt}
    \vspace*{-4mm}
\caption{Nonempty State Type}
\label{pvs:empty}
\end{figure}

\myparagraph{Updating the Scaffold.}
The \texttt{good\_entry?}  predicate is suitable for characterizing
the entry-wise correctness of a scaffold, but a
stronger predicate \texttt{fine\_entry?} is needed to check an entry that
will be used to update the scaffold.  Note that while the scaffold is initalized
with \texttt{pending} entries, they are never actually inserted into the scaffold.  All updates either finalize
a parse for a nonterminal at a given input position to a \texttt{loop},
\texttt{fail}, or \texttt{good} entry, or push a new entry into the stack.
We will be defining a function to compute such an entry for the current stack head consisting of position \texttt{i} and nonterminal \texttt{n}\@.  This function
computes an entry \texttt{u} which is either a finalized entry for
\texttt{A(i)(n)} or of the form \texttt{push(i', n')}
indicating that the new stack head will be at $(\mathtt{i'}, \mathtt{n'})$, and
the old stack $(\mathtt{i}, \mathtt{n})$ is pushed into this position, i.e., we set
\texttt{A(i')(n')} to \texttt{push(i, n)}\@.  
The \texttt{fine\_entry?} predicate on the newly computed entry \texttt{u}
is a strengthened version of  \texttt{good\_entry?} which checks that 
\texttt{u} is either a \texttt{fail}, \texttt{loop}, or
\texttt{good} entry according to \texttt{good\_fail?}, \texttt{loop\_ready?}, or \texttt{good\_good}, respectively, or it is of the form \texttt{push(i', n')}, where
\texttt{n'} is a nonterminal and the parent entry \texttt{A(i')(n')} in the stack
is either a \texttt{push} or a \texttt{pending} entry.  The case where \texttt{A(i')(n')} is \texttt{pending} arises because the parse at $(\mathtt{i', n'})$ is a successor of $(\mathtt{i, n})$ that has not yet been parsed.  The case where
\texttt{A(i')(n')} is a \texttt{push} entry arises when the successor node $(\mathtt{i', n'})$ to $(\mathtt{i, n})$ is already on the stack.  
The strengthened condition introduced in \texttt{fine\_entry?}
only became apparent through failed proof attempts.  Experimenting with the proof is an effective way to explore how the complex web of relationships on the scaffold are affected by any updates.  

\begin{figure}[h!]
  \lstinputlisting[language=PVS]{code_samples/fineentry.txt}
    \vspace*{-4mm}
\caption{Condition on an entry to be inserted into the scaffold}
\label{pvs:fineentry}
\end{figure}

It turns out that the definition of \texttt{fine\_entry?} is also not
strong enough.  This is because if the scaffold entry
\texttt{A(i')(n')} is a \texttt{push} entry, then the parser would
have detected this loop.  Therefore, the only case when \texttt{u} is
of the form \texttt{push(i', n')} is when \texttt{A(i')(n')} is
\texttt{pending}.  We also needed the explicit condition that
$(\mathtt{i', n'})$ is a successor of the top of the stack $(\mathtt{i,
  n})$\@.  This strengthening is also a consequence of failed proof
attempts and is captured in the definition of the predicate
\texttt{finer\_entry?}\@.  However, \texttt{fine\_entry?} is independently
useful in the proof so we retain the definition of \texttt{fine\_entry?}
instead of inlining it into that of \texttt{finer\_entry?}\@.  

\begin{figure}[h!]
  \lstinputlisting[language=PVS]{code_samples/finerentry.txt}
    \vspace*{-4mm}
\caption{Strengthened condition on an entry to be inserted into the scaffold}
\label{pvs:finerentry}
\end{figure}

The predicate \texttt{finer\_entry?} can be used for a subtype definition
for \texttt{finer\_entry}\@.
\begin{figure}[h!]
  \lstinputlisting[language=PVS]{code_samples/finerentrytype.txt}
    \vspace*{-4mm}
\caption{Predicate subtype declaration for \texttt{finer\_entry}}
\label{pvs:finerentrytype}
\end{figure}



We again emphasize that the predicates on scaffold entries defined
above are neither tedious nor obvious.  They are crucial to the
characterization of the correctness of the parser as well as to the
construction of invariants that are preserved during parsing.  They
are constructed hand-in-hand with the definition of the parser through
failed proof attempts and repeated refactoring.  This iterative
process is fairly painless.  It usually takes only a few minutes to
converge on the failing cases of the proof to discover where a
definition is too strong or too weak or simply incorrect.  The
insights learned from these failing proofs help focus human attention
on the tricky points in both the algorithms and the proof.

% One might argue that these false starts could be avoided by better
% planning.  Unfortunately, the case analyses in these proofs are
% extremely complex and delicate.  Work through them without the
% benefit of automation would be painfully tedious and error-prone.  

We finally arrive at the meat of the parser,  the function \texttt{newentry} which computes a new entry for updating the scaffold, the one for which we
spelled out the \texttt{finer\_entry} type.  The \texttt{newentry} function operates on the grammar \texttt{G}, the input string \texttt{s}, the original start position \texttt{start}, the root nonterminal \texttt{rootnt}, and the current
state \texttt{St}\@.  The return type for the new entry is defined
by the \texttt{newentry} dependent type.   The definition first
extracts the top position \texttt{pos} and \texttt{nonterminal} \texttt{cur}
of the stack, and uses record field access (backquote) to 
look up the tail of the stack as \texttt{St`scaf(pos)(cur)}. 
The definition has eight cases according to the grammar entry \texttt{G(cur)}\@.
When \texttt{G(cur)} is \texttt{epsilon}, the new entry is \texttt{good(0, 0)},
indicating depth $0$ and span $0$ for the parse.  If \texttt{G(cur)} is \texttt{failure}, then the new entry is \texttt{fail(0)} indicating a nesting depth of $0$\@.
When \texttt{G(cur)} is \texttt{any(p)}, result is
\texttt{fail(0)} if either the position \texttt{pos} is \texttt{len}
  or the string character \texttt{s(pos)} does not satisfy the predicate \texttt{p}\@.  Otherwise, the result is \texttt{good(0, 1)} indicating a nesting depth of $0$ and a span of length $1$\@.  
The case when \texttt{G(cur)} is \texttt{terminal(a)} is similar.
The remaining cases are handled separately and described below. 

\begin{figure}[h!]
  \lstinputlisting[language=PVS]{code_samples/newentrytop.txt}
    \vspace*{-4mm}
\caption{Top level of the \texttt{newentry} parsing function}
\label{pvs:newentrytop}
\end{figure}


When \texttt{G(cur)} is \texttt{concat(n1, n2)}, we look up the entry \texttt{u1}
given by \texttt{St`scaf(pos)(n1)}\@.   If \texttt{u1} is of the form \texttt{fail(d1)}, then
the result is \texttt{fail(d1 + 1)}\@.  If \texttt{u1} is \texttt{good(d1, sp)}, then
we lookup \texttt{u2} given by \texttt{St`scaf(pos + sp)(n2)}\@.  If
\texttt{u2} is \texttt{fail(d2)}, then the result is \texttt{fail(1 + max(d1, d2))} to capture the increment to the nesting depth from both d1 and d2.
If \texttt{u2} is \texttt{good(d2, sp2)}, then the result is
\texttt{good(1 + max(d1, d2), sp + sp2)} since the parse for \texttt{cur} at position \texttt{pos} is obtained by concatenating the parses for n1 at \texttt{pos}, and
\texttt{n2} at \texttt{pos + sp}\@.  When \texttt{u2} is \texttt{pending},
we need to launch a parse query for \texttt{n2} at position \texttt{pos + sp},
and this is done by returning the entry \texttt{push(pos + sp, n2)}\@.
The only remaining possibilities are that \texttt{u2} is \texttt{loop} or
of the form \texttt{push(i, n)}\@.  In either case, the result is \texttt{loop}
since we have a previously detected loop at \texttt{pos + sp}, \texttt{n2},
or we have detected a new loop since \texttt{pos + sp}, \texttt{n2} is already
on the stack.  The remaining cases of the definition for \texttt{u2}
are similar to ones described for \texttt{u2}\@.  
\begin{figure}[h!]
  \lstinputlisting[language=PVS]{code_samples/newentry_concat.txt}
    \vspace*{-4mm}
\caption{\texttt{newentry} definition for \texttt{concat(n1, n2)}}
\label{pvs:newentry_concat}
\end{figure}

The definition in Figure~\ref{pvs:newentry_choice}
for the case when \texttt{G(cur)} is \texttt{choice(n1, n2)}
is quite similar.
\begin{figure}[h!]
  \lstinputlisting[language=PVS]{code_samples/newentry_choice.txt}
    \vspace*{-4mm}
\caption{\texttt{newentry} definition for \texttt{choice(n1, n2)}}
\label{pvs:newentry_choice}
\end{figure}

We omit the definitions for the case where \texttt{G(cur)} is either \texttt{check(n1)}
or \texttt{neg(n1)} since these are similar to those we have already described. 

\begin{figure}[h!]
  \lstinputlisting[language=PVS]{code_samples/newentry_check.txt}
    \vspace*{-4mm}
\caption{\texttt{newentry} definition for \texttt{check(n1)}}
\label{pvs:newentry_check}
\end{figure}

\begin{figure}[h!]
  \lstinputlisting[language=PVS]{code_samples/newentry_neg.txt}
    \vspace*{-4mm}
\caption{\texttt{newentry} definition for \texttt{neg(n1)}}
\label{pvs:newentry_neg}
\end{figure}

\myparagraph{The Step Function.}
The \texttt{newentry} function forms the core of the step function \texttt{step}
for our parsing state machine defined in Figure~\ref{pvs:step}\@.  The state machine halts when
the stack component of the state is empty.  Otherwise,
we compute the \texttt{newentry} using the state \texttt{St}.
If \texttt{newentry} is of the form \texttt{push(p, n)}, then
we push this entry into the stack.  Otherwise, \texttt{newentry}
is a finalized parse entry that we use to update the scaffold, and pop the stack.
The loop detection flag \texttt{lflag} is set when \texttt{newentry} is \texttt{loop}.  The step function \texttt{step} is the body of a tail recursive \texttt{parse} function.  The termination measure for the parse operation is a double lexicographic
value defined by \texttt{size} (whose definition we omit):
either the number of non-finalized cells in the scaffold decreases,
or it remains the same and the stack size grows up to a bound.  
\begin{figure}[h!]
  \lstinputlisting[language=PVS]{code_samples/step.txt}
    \vspace*{-4mm}
\caption{The parsing state machine step function}
\label{pvs:step}
\end{figure}


% \begin{figure}[h!]
% \lstinputlisting[language=PVS]{code_samples/size.txt}
% \caption{The termination measure for the parser}
% \label{pvs:size}
% \end{figure}

The state returned by the parser is always empty.  
\begin{figure}[h!]
  \lstinputlisting[language=PVS]{code_samples/endstate.txt}
    \vspace*{-4mm}
\caption{The termination state}
\label{pvs:endstate}
\end{figure}


\myparagraph{The Parser.}
The parse function \texttt{parse} shown in Figure~\ref{pvs:parser}
invokes the step function until
the stack is empty, resulting in an empty state of 
\texttt{endstate} type  (whose definition is omitted).  
\begin{figure}[h!]
  \lstinputlisting[language=PVS]{code_samples/parser.txt}
    \vspace*{-4mm}
\caption{The parsing engine}
\label{pvs:parser}
\end{figure}
The gateway function to the parser \texttt{doparse} is defined in Figure~\ref{pvs:doparse} to initialize
the state and invoke the \texttt{parser} operation\@.
\begin{figure}[h!]
  \lstinputlisting[language=PVS]{code_samples/doparse.txt}
    \vspace*{-4mm}
\caption{The gateway parsing operation}
\label{pvs:doparse}
\end{figure}

\subsection{Proving Correctness}
We now discuss the correctness proof.  The correctness criteria are
essentially embedded in the type signatures of the operations.
For example, the definition of the \texttt{endstate} type ensures that
all the entries in the scaffold are well-formed, i.e., satisfy the chart constraints for the given grammar.   Since the stack is empty, all the loops are actual
loops, and the root query has a finalized entry.  There are two ways of
capturing correctness through the type signature.  One approach uses the declared
type signature of the operation as has been done with \texttt{newentry},  
\texttt{parse}, and \texttt{doparse}\@.  The other approach employs the
typing judgement mechanism in PVS to introduce a new type signature for
already declared operations and expressions.  We have only made sparing
use of typing judgements here since both the basic well-formedness of the
parsing algorithm and termination rely heavily on the invariants used in
the correctness arguments.

The correctness proof is constructed by discharging the subtyping and
termination type correctness conditions (TCCs) which are the proof
obligations generated by the PVS typechecker.  The newentry operation
along generates fifty nine TCCs.  A few of these are mostly trivial
bounds constraints on those parameters that are declared as
\texttt{uint8}, \texttt{uint32}, or \texttt{uint64}, but even in some
of these cases, the proofs are nontrivial. Twenty eight of the
fifty nine TCCS for \texttt{newentry} are those verifying that the
entry \texttt{u} constructed by each of the twenty eight cases of the
definition of \texttt{newentry} satisfies the \texttt{finer\_entry?}
predicate.  The proofs, which are highly reusable, combine 
a few key lemmas and heavy use of the built-in satisfiability modulo
theories (SMT) decision procedures.  One such key lemma
shown in Figure~\ref{pvs:goodgoodtscaffold} captures how the
\texttt{good\_tscaffold} condition is preserved through an update
with a new entry.  
\begin{figure}[h!]
  \lstinputlisting[language=PVS]{code_samples/goodgoodtscaffold.txt}
    \vspace*{-4mm}
\caption{\small Key lemma for invariant preservation}
\label{pvs:goodgoodtscaffold}
\end{figure}



PVS proofs are robust since they rely on a high degree of guided
automation.  They can be easily edit and reused on similar proof
obligations, or transformed into proof strategies that are effective
on a broad class of proof goals.  The flexibility in the level of
automation and transparency allows users to interact when needed in
order to diagnose and fix problematic definitions, theorem statements,
and proofs.  User attention is also needed to identify patterns that
are common to the proofs so that they can be encapsulated in
refactored definitions, new lemmas, and proof strategies.  Iteratively
developing the definitions, types, and lemmas from scratch would 
take on the order of three to four weeks.  For example,
developing the lemmas and proofs for discharging the TCCs for the flat
version of the parser took around week, whereas the refactored version
using \texttt{newentry} required only a couple of days since it reused
the proofs from the first attempt.  The proof required seventeen
supporting lemmas and only a handful of these needed nontrivial
interaction.  A finite-model counterexample capability similar to that
in Alloy~\cite{Jackson2006:book}, Ivy~\cite{padon2016ivy}, or
Nitpick~\cite{blanchette2010nitpick} might be useful but not essential
since it is typically quite easy to diagnose the issue from a failed
proof branch.




\section{Conclusions}

The present exercise is a step in the direction of building proof
infrastructure for the robust and efficient verification of practical
parsing algorithms.  We have successfully tested the parser on
several examples (including those shown in Section~\ref{sec:PEG}) using the
PVS2CL Common Lisp code generator and the PVSio read-eval-print loop.
We plan to continue polishing the proof and extending it to PEGs with
data-dependent grammars.  %anonymized~\cite{mundkurparsley}.

We started our formalization attempt on a bottom-up, right-to-left
scaffolding automaton parser where the scaffold entries are filled in
by scanning the scaffold from the last input position to the start
position, and scanning the nonterminals at each input position.  A
nonterminal at a given position might trigger a sub-query on another
(or the same) nonterminal at the same position whose entry has not yet
been finalized.  We could perform a static analysis on the grammar in
order to pick an ordering that ensured that there are no parsing loops
and sub-queries to each query are finalized ahead of the query.
Another option is to use the nonterminals in their numeric order while
detecting cycles dynamically.  We chose the second approach since
there is no effective check for whether a PEG is well-formed, and loop
detection poses an interesting verification challenge.  For
dynamic loop detection, one could repeatedly scan the nonterminals at
each position in the scaffold until a fixpoint has been reached so
that any entries that have not been finalized can be classified as
loops.  Alternately, one could use a hybrid approach of combining a
scan with a stack of queries for those sub-queries that have not yet
been finalized. The stack can also be used for loop detection.

If we are going to employ a query stack, then a top-down parser makes
more sense since it does not compute irrelevant scaffold entries.  We
defined a top-down parser with state that contained both the scaffold
and an explicit stack, and quickly observed that the stack could be
built into the scaffold data structure.  This yields a significant
efficiency in loop detection: an entry lookup versus a scan
through an external stack. That led to the top-down chart parser
described in this paper.  After some preliminary experiments, we
arrived at the definition of the \texttt{state} type and the
dependency ordering of the fields within the state type.  We first
defined a flat \texttt{step} function which we could then refactor by
introducing the \texttt{newentry} definition and discharging the
resulting proof obligations.  There were some subtleties with the type
signature for the \texttt{newentry} function since it constructs two
kinds of entries: the finalized ones and the new queries/stack entries.  We did
succeed in finding an elegant uniform characterization of the type
signature using the \texttt{finer\_entry?} predicate.  By supporting
efficient proof exploration and engineering, a proof assistant
leverages the power of typechecking, proof construction and certification, and
code generation to serve as an effective problem-solving aid.  

The proofs we have completed are by no means the last word.  There is definitely room to squeeze out more robustness and automation through the introduction of
auxilliary functions, lemmas, typing judgements, and customized proof strategies.
Eventually, we hope that verifying a complex parsing algorithm will be a
largely automatable task as we understand how the parsing metatheory can be
encapsulated into types and proof strategies.  We also plan to extend the basic
PEG grammar with inherited and synthesized attributes, and constraint checking,
to capture some of the features the Parsley data format description language.  

The present exercise is a step in the direction of building proof
infrastructure for the robust and efficient verification of practical
parsing algorithms.  The definitions have been written in a form that
is friendly to generating efficient C code.  Since we are still polishing
the proof, we have not currently had the time to thoroughly test the
generated C code.  We have successfully tested the parser using the
PVS2CL Common Lisp code generator and the PVSio read-eval-print loop.

%\section{Conclusions}
\clearpage
\balance
\bibliography{biblio.bib}


\end{document}